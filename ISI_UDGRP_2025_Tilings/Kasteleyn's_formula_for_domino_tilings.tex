\documentclass{beamer}
%theme
\usetheme{Berlin}
\usecolortheme{default}
\setbeamertemplate{theorems}[numbered]
\setbeamertemplate{caption}[numbered]
%packages
\usepackage[utf8]{inputenc}
\usepackage{graphicx,physics,amssymb,amsmath,amsthm}
\usepackage[style=authoryear, backend=biber]{biblatex}
\usepackage{tikz}
\usetikzlibrary{matrix}
\addbibresource{references.bib}
%Numbering
\numberwithin{theorem}{section} 
\theoremstyle{definition}
\newtheorem{remark}{Remark}
%starts here
\title{Kasteleyn's formula for domino tilings of rectangles}
\subtitle{- presentation on Eric Stucky's expository paper -}
\author{Arkaraj Mukherjee}
\date{\today}

\begin{document}

\begin{frame}
    \maketitle
\end{frame}

\begin{frame}
    \frametitle{Contents}
    \tableofcontents
\end{frame}

% ==========================================
% SECTION 1: Overview
% ==========================================
\section{Overview}
\begin{frame}
    This is a presentation on Kasteleyn's formula for the number of domino tilings of a $m\times n$ rectangle heavily based on [\cite{stucky2015}]. 
    Although there are many connections between this and physics, we will not cover these here and instead direct the interested readers to [\cite{kasteleyn1961}]. 
    The key result we prove is : If $T(m,n)$ is the set of tilings of a $m\times n$ rectangle (wlog. $m$ is even) and $H(\pi),V(\pi)$ the number of horizontal and vertical dominos in $\pi\in T(m,n)$ respectively, then for any complex $x,y$ we have
    $$\sum_{\pi\in T(m,n)}x^{H(\pi)}y^{V(\pi)}=2^{\frac{mn}{2}}\prod_{k=1}^{m/2}\prod_{l=1}^{n}\qty(x^2\cos^2\frac{\pi k}{m+1}+y^2\cos^2\frac{\pi l}{m+1})^{\frac{1}{2}}$$
    If we set $x=y=1$ this gives us the number of such tilings.
\end{frame}
\begin{frame}
    The proof is seperated into two major parts. 
    In the first we deal with the combinatorial aspect where we translate the problem of counting tilings into that of counting specific permutations allowing us to use matrix methods\footnote[1]{
    We have already seen things like the State matrix in the same context, tilings of hexagons with lozenges using determinants etc.
    } especially Pfaffians which deem to be just what we need here. 
    For a specific $mn\times mn$ matrix $D$, the expression in the main theorem will be $\operatorname{Pf}(D)$. 
    We rely on the identity $\operatorname{Pf}(K)^2=\det(K)$ where $K$ is skew symmetric of even dimension. 
    First we will come up with a \textit{guess} $D'$, which only differs from $D$ by some signs but a hardest problem of this part is constructing $D$ itself, signs included.
    In the second part of the proof we will use kronecker products and (...left out till I read this...)
\end{frame}

% =========================================
% SECTION 2: Bijection with some specific permutations
% =========================================

\section{Translation into counting permutations}

\begin{frame}
    As discussed above we will try to connect tilings with dominos to permutations on $[mn].$ 
    We will identify the usual unit cells in the $m\times n$ rectangle with numbers from $[mn]$ using the lexicographic ordering, starting from the bottom left corner. 
    For example, when $m=n=2$ we identify, 
    $$(1,1)\leftrightarrow 1,(1,2)\leftrightarrow 2,(2,1)\leftrightarrow 3,(2,2)\leftrightarrow 4$$
    Now we line all the dominos up and label both of their ends with adjacent numbers as follows: Label the ends of the first domino with $1$ and $2$, the second with $3$ and $4,\ldots$ upto $mn/2$ dominos. 
    Then, we can associate to each tiling a permutation $\sigma$ on $[mn]$ which maps $k$ (some end of a domino) to the lexicographic rank of its position on the $m\times n$ rectangle. 
\end{frame}
\begin{frame}
    It is easy to notice that a tiling can have multiple permutations corresponding to it (this is fixed over all tilings, finding how many corresponding permutations there are is a good exercise) and also that not every permutation corresponds to a tiling (if this weren't the case then this problem would be trivial), for example take $(2 4)$ on $[4]$. 
    We seek some way to identify an unique permutation corresponding with some tilings, something \textit{canonical}. 
    One way to do this is to put these restrictions on the permutations : $\sigma(2k-1)<\sigma(2k)$ and $\sigma(1)<\sigma(3)<\sigma(5)<\ldots$. 
    These hand-wavingly make sense because flipping a domino should be immaterial to the number of tilings so its good to fix the higher label(equivalently both) and also that switching \textit{pairs} around just replaces two dominos which should also be immaterial - for example, on $[4]$ we see that $1234$ and $3412$ correspond to the exact same tiling.
    Convince yourself that such permutations exist given any tiling, leaving uniqueness for exercise.
\end{frame}

% =========================================
% SECTION 3: Constructing the matrix 
% =========================================

\section{Constructing the matrix}

\begin{frame}
    We first try to get a glimpse of the possible structure of $D$, only then rectifying our guess to the final matrix $D$ which will give us $\operatorname{Pf}(D)=|T(m,n)|$. 
    The Pfaffian goes through all the permutations which could correspond to some tiling, but we want a way to filter out the \textit{bad} permutations.
    A way to do this is by setting all the entries $D_{i,j}$ where $i,j$ can not be labels on a singular domino in some cannonical permutation, to zero.
    Then the only non zero entries could be :
    \begin{enumerate}
        \item horizontal dominos: clearly the larger even label must lie exactly to the right of the smaller odd label, unless the odd label is on the rightmost edge, in which case it can not correspond to a domino and we set it to zero.
        \item vertical dominos: here, the even label is similarly always directly above the smaller odd one, unless this odd one is on the top most edge.
    \end{enumerate}
\end{frame}

\begin{frame}
    Thus combining these with the definition of permutations we are interested in, we take the following guess at $D$, namely $D'$: the $(i,j)$-th entry is nonzero if 
    \begin{itemize}
        \item $i$ is odd, $j=i+1$ and $m\not|i$ (not on the right edge basically.)
        \item $j=i+m$ (directly above) and $i\leq mn-m$ (not on the topmost row).
    \end{itemize}
    Now if we make the entries which correspond to horizontal dominos $x$ and the vertical ones $y$ respectively, then
    $$\prod_{i=1}^{mn/2}D_{\sigma(2i-1),\sigma(2i)}=
    \begin{cases}
        x^{H(T)}y^{V(T)}\text{ if }\sigma\text{ corresponds to some tiling } $T$\\
        0\text{ otherwise} 
    \end{cases}$$
    thus, using the notation developed prior to this we have
    $$\operatorname{Pf}(D')=\sum_{\pi\in T(m,n)}\operatorname{sgn}(\sigma_\pi)x^{H(\pi)}y^{V(\pi)}$$
\end{frame}

\begin{frame}
    This was a very good guess but it doesn't work as a sign is multiplied to every summand.
    If we want to alter $D'$ and make it work, then the most natural course of action is to analyse $\operatorname{sgn}(\sigma_\pi)$ for tilings $\pi$. 
    It is proved in [\cite{stucky2015}] that the sign of the permutation corresponding to a tiling is the number of dominos inside odd columns, thus we can nullify the sign upfront by assigning $-y$ to the entries corresponding to these dominos, instead of $y$. 
    The proof is totally elementary but best explained with visuals which I do not feel the need to reproduce here so interested readers can see it for themselves in [\cite{stucky2015}]. Compiling everything we know about $D$ till now, heres a full description of the nonzero entries:
    \begin{itemize}
        \item $D_{i,i+1}=x$ and $D_{i+1,i}=-x$ for $1\leq i\leq mn$ such that $m\not|i$.
        \item $D_{2i,2i+m}=y$ and $D_{2i+m,2i}=-y$ for $1\leq i\leq m(n-1)/2$
        \item $D_{2i-1,2i-1+m}=-y$ and $D_{2i-1+m,2i-1}=y$ for $2\leq 2i\leq m(n-1)+1\iff 1\leq i\leq m(n-1)/2$
    \end{itemize}
\end{frame}
\begin{frame}
    With this construction we see that $D$ is skew symmetric and
    $$\operatorname{Pf}(D)=\sum_{\pi\in T(m,n)}x^{H(\pi)}y^{V(\pi)}$$
    all that is left for us to do now is grind through the algebra and calculate $\det(A)$ as its directly related to the pfaffian. 
\end{frame}

% =========================================
% SECTION 4: The Pfaffian
% =========================================

\section{The Pfaffian}

\begin{frame}
    \begin{definition}[Pfaffian]
        For a $2n\times 2n$ skew symmetric matrix $A$, the pfaffian is defined as
        $$\operatorname{Pf}(A):=\frac{1}{2^nn!}\sum_{\sigma\in S_{2n}}\operatorname{sgn}(\sigma)\prod_{i=1}^na_{\sigma(2i-1),\sigma(2i)}$$
    \end{definition}
    However, there is a way to write the pfaffian with lesser terms. Let $\Pi$ be the set of partitions of $[2n]$ into pairs.
    For such a partition $\pi=\{\{i_1,j_1\},\ldots,\{i_n,j_n\}\}$ where $i_1<i_2<i_3<\ldots$ and $i_k<j_k$ for all $k$, let $\sigma_\pi$ be the permutation $i_1j_1i_2j_2\ldots$.
    With the notation above, we claim that
    $$\operatorname{Pf}(A)=\sum_{\pi\in\Pi}\operatorname{sgn}(\sigma_\pi)\prod_{i=1}^na_{\sigma_\pi(2i-1),\sigma_\pi(2i)}$$
\end{frame}
\begin{frame}
    For a proof, notice that when the pairs in the partition are permuted and the elements in each pair also permuted, these correspond to $2^nn!$ permutations and that the summand remains constant upon switching position of two elements in a pair as $A$ is skew symmetric and it stays constant over as the sign, being a homomorphism on $S_{2n}$, leads is multiplied by 
    $$\operatorname{sgn}((2i-1,2j-1)(2i,2j))=\operatorname{sgn}((2i-1,2j-1))\cdot\operatorname{sgn}((2i,2j))$$
    $$=(-1)\cdot(-1)=1$$
    in cycle notation, when we switch the $i$-th and $j$-th pairs - making the summand stay constant all over these permutations and we are done.
\end{frame}
\begin{frame}
    \begin{theorem}[spectral theorem for real skew symmetric matrices]
        If $A$ is a $2n\times 2n$ real skew symmetric matrix then the spectrum is of form $\{\pm i\lambda_k\}_{1\leq k\leq n}$ and $A$ is orthogonally similar to a block diagonal matrix $H$ with diagonal entries being $H_1,\ldots,H_n$ with
        \begin{center}
            $H_k:=\begin{pmatrix}
                0 & \lambda_k\\ -\lambda_k & 0
            \end{pmatrix}$
        \end{center}
    \end{theorem}
    \begin{proof}
        The structure of the spectrum is a result from basic linear algebra. 
        As skew symmetric matrices are normal we can diagonalise them into $D:=Q^{\top}AQ=\operatorname{diag}(i\lambda_1,-i\lambda_1,\ldots)$ with $Q$ orthogonal. Let,
        \begin{center}
            $P:=\frac{1}{2}\cdot\begin{pmatrix}
            1 & 1 \\ i & -i
        \end{pmatrix}$
    \end{center}
    \end{proof}
\end{frame}
\begin{frame}
    \begin{proof}[(continued)]
        This is an orthogonal matrix. Let $L=\operatorname{diag}(P,P,\ldots)$ be a $2n\times 2n$ block diagonal matrix, this is also orthogonal and we see that
        $$L^\top HL =\operatorname{diag}(P^{\top},P^{\top},\ldots)\operatorname{diag}(H_1,H_2,\ldots)\operatorname{diag}(P,P,\ldots)$$
        $$=\operatorname{diag}(P^\top H_1 P,P^\top H_2 P,\ldots)=\operatorname{diag}(\operatorname{diag}(i\lambda_1,-i\lambda_1),\ldots)=D$$
        Thus, $A=Q^{\top}L^{\top}HLQ$ rearranging which we can conclude as the product of two orthogonal matrices is orthogonal.
    \end{proof}
    \begin{theorem}
        For any $2n\times 2n$ skew symmetric matrix $A$ and arbitrary matrix $B$ we have, $\operatorname{Pf}(BAB^\top)=\operatorname{Pf}(A)$.
    \end{theorem}
\end{frame}
\begin{frame}
    For our use case however $B$ is invertible and this version of the theorem can be proved via considering $B$ to be a product of elementary matrices and using some of the determinant-like properties which can be found on wikipedia. 
    Now we state the most important theorem we will use regarding pfaffians.
    \begin{theorem}[T. Muir. 1960]
        If $A$ is $2n\times 2n$ and skew symmetric then $\operatorname{Pf}(A)^2=\det(A)$
    \end{theorem}
    \begin{proof}
        Show that $P\in\mathbb R[x_1,\ldots,x_m]$ and $P(\mathbb R^m)=\{0\}\implies P\equiv 0$ as an exercise.
        Thus it suffices to prove this theorem for all real matrices.
        We have seen that we can write $A=QHQ^\top$ with $H$ the same as above, for some $Q$. 
        Then, $\operatorname{Pf}(A)^2=\operatorname{Pf}(H)^2\det(Q)^2=\det(A)^2$ with the second last equality left as another exercise.
    \end{proof}
\end{frame}

% =========================================
% SECTION 5: representing the matrix using kronecker products 
% =========================================

\section{Representing the matrix with kronecker products}

\begin{frame}
    \begin{definition}[kronecker product or, tensor product\footnote{There are connections aplenty between this and the conventional Tensor products but here this is used interchangably with the kronecker product}]
        If $A$ is $p\times q$ and $B$ is $m\times n$, we define $(A\otimes B)_{ij}:=a_{ij}\cdot B$ in block form. This is a $mp\times nq$ matrix.
    \end{definition}
\end{frame}

% =========================================
% SECTION ?: References
% =========================================
\section{References}

\begin{frame}[allowframebreaks]
    \frametitle{References}
    \printbibliography
\end{frame}

\end{document}